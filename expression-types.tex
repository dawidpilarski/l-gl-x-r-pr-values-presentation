\documentclass[10pt]{beamer}

\usetheme[progressbar=frametitle, block=fill]{metropolis}
\usepackage{xcolor}
\usepackage{multirow}
\usepackage{pgfpages}
\usepackage{pifont}
\usepackage[utf8]{inputenc}
\usepackage[T1]{fontenc}
\usepackage[english]{babel}
\newcommand{\cmark}{\ding{51}}
\newcommand{\xmark}{\ding{55}}
\setbeamertemplate{note page}{\insertnote}
%\setbeameroption{show notes on second screen=left}
\setbeameroption{hide notes}
\definecolor{amethyst}{rgb}{0.5, 0.4, 1.0}
\definecolor{amethystgrey}{rgb}{0.85, 0.85, 1.0}
\definecolor{amethystdark}{rgb}{0.4, 0.3, 0.9}
\definecolor{orangedark}{rgb}{0.0, 0.9, 0}
%\definecolor{titlebg}{HTML}{4e8074}
%\definecolor{titlebg}{HTML}{3e7985}
\definecolor{titlebg}{HTML}{fbf8ff}
\definecolor{font}{HTML}{23373b}
%\setbeamercolor{title}{fg=amethyst, bg=amethyst}
\setbeamercolor{frametitle}{fg= font, bg=titlebg}
%\setbeamercolor{section title}{black}
%\setbeamercolor{structure}{fg=amethyst, bg=amethyst}
\setbeamercolor{progress bar}{ fg = amethyst, bg= amethystgrey }
%\setbeamercolor{itemize item}{fg=amethyst,bg=white}
\setbeamercolor{alerted text}{fg=amethystdark}
%\setbeamercolor{title separator}{ ... }
%\setbeamercolor{progress bar in head/foot}{ ... }
%\setbeamercolor{progress bar in section page}{ ... }

\usepackage{booktabs}
\usepackage[scale=2]{ccicons}

\usepackage{minted}

\usepackage{pgfplots}

\usepackage{xspace}

\title{[l, gl, x, r, pr]values}
\subtitle{Value categories}
\date{}
\author{Dawid Pilarski}
\institute{dawid.pilarski@tomtom.com}

\begin{document}

\maketitle

\section{Introduction}

\begin{frame}{How are expressions categorized?}
\centering
	\includegraphics[width=\linewidth]{value_categories-1.jpg}
\end{frame}

\begin{frame}{How to understand fundamental classifications?}
	\begin{itemize}
		\item lvalue - T\& \pause
		\item xvalue - T\&\& \pause
		\item prvalue - T
	\end{itemize}
\end{frame}

\begin{frame}{The common mistake}
	\centering
	Usually people think about expression categories: \\ 
	\vskip 3em
	\includegraphics[width=0.5\linewidth]{value_categories-1.jpg} \\
		\vskip 3em As categories of references, which is \alert{\color{red}wrong}
\end{frame}

\begin{frame}{Getting it right}
	\centering
	$$ category <=> expression $$
	$$ reference => category $$
	$$ category \ne > reference$$
	
	\vskip 3em
	
	\footnotesize
	[Note: there is no reference of type prvalue]
\end{frame}

\begin{frame}{prvalue vs glvalues}
	\begin{description}
		\item[glvalues] \hfill \\ 
			Generalized lvalues. It's everything that \alert{references the \emph{object}}
		\item[prvalues] \hfill \\
			Pure rvalues. It's a \alert{value}.
	\end{description}
\end{frame}

\begin{frame}{Values vs Objects}
	\begin{columns}[t]
		\begin{column}{0.45\linewidth}
			\centering{Objects}
			\vskip 0.5em
			\hrule
			\begin{itemize}
				\item many object with same value
				\item object can be changed
				\item many references to the same object
			\end{itemize}
		\end{column}
		\begin{column}{0.45\linewidth}
			\centering{Values}
			\vskip 0.5em
			\hrule
			\begin{itemize}
				\item value is unique
				\item value cannot be changed
				\item value
			\end{itemize}
		\end{column}	
	\end{columns}
\end{frame}

\section{Into the details - glvalues}

\begin{frame}{Xvalues}
	xvalues mean: \hfill \\ \hskip3em \alert{eXpiring values}

	\vskip3em

	Xvalues are such kind of expressions, that its' results point to the object,
	which will soon \alert{expire}.

\end{frame}

\begin{frame}{Xvalues examples}
	There are fixed number of ways we can get xvalues:
	\begin{itemize}
		\item function call which result type is rvalue reference (T\&\&).
		\item explicit cast to rvalue reference.
		\item subscript operator call on the xvalue arrays.
		\item non reference member access to the xvalue objects (also through pointer to member).
		\item temporary materialization conversion.
	\end{itemize}
\end{frame}

\begin{frame}[fragile]{function call which result type is rvalue reference}
	\begin{minted}{c++}
struct Foo{};
Foo&& bar(); 

int main(){
  bar(); // "bar()" is the xvalue expression 
}
	\end{minted}
\end{frame}

\begin{frame}[fragile]{explicit cast to rvalue reference}

\begin{minted}{c++}
struct Foo{/* definition */};

int main() {
  Foo a;
  std::move(a); // "std::move(a)" casts a to Foo&&
  static_cast<Foo&&>(a); // does same thing as std::move
}
\end{minted}

\end{frame}

\begin{frame}[fragile]{subscript operator call on the xvalue arrays}
	\begin{minted}{c++}
int main(){
  Foo arr[10] = {};
  std::move(arr)[0]; // xvalue ref to the first arr element
}
	\end{minted}
\end{frame}

\begin{frame}[fragile]{non reference member access  to the xvalue objects}
	\begin{minted}{c++}
template <typename T>
struct Foo{
T member;
};

int main(){
  Foo<int> a{};
  std::move(a).member; //xvalue 
  
  Foo<int&> a{.member = a.member};
  std::move(a).member; // lvalue 
                       // due to reference collapsing
}
	\end{minted}
\end{frame}

\begin{frame}[fragile]{non reference member access  to the xvalue objects II}
	\begin{minted}{c++}
int main(){
  int Foo<int>::* pointer = &Foo<int>::member;
  Foo<int> foo{};
  std::move(foo).*pointer; //xvalue expression
  return 0;
}
	\end{minted}
\end{frame}

\begin{frame}[fragile]{temporary materialization conversion}
	\begin{minted}{c++}
struct Foo{int member;};
Foo().member; // member access requires glvalue
              // tmc converts the prvalue to xvalue
	\end{minted}
\end{frame}


\begin{frame}[fragile]{Complete type requirements}
	\centering
	glvalue expressions can operate on non-complete type
	\vskip 0.5em
	\hrule
	
	\begin{minted}{c++}
struct Foo{};

Foo& first_foo();
Foo& second_foo();

Foo& first_of_two(Foo& first, Foo& second){return first;}

int main(){
  auto& result = first_of_two(second_foo(), first_foo());
  if(&result == &second_foo())
    std::cout << "result is second" << std::endl; 
}
		
	\end{minted}
\end{frame}

\begin{frame}{glvalue and void}
	expression, which result is of type void cannot be glvalue expression.
	\vskip 2em
	
	\begin{itemize}
		\item It's impossible to create object of type void
		\item It's impossible to have a reference to void
	\end{itemize}
\end{frame}

\section{into the details - prvalues}

\begin{frame}{What are prvalues expressions}
	Those are expression which results are the \alert{values}.
\end{frame}

\begin{frame}[fragile]{prvalues examples}
	\begin{minted}{c++}
	struct Foo{};
	Foo(); // returns value of type Foo.
	
	Foo bar();
	bar(); // prvalue returns type Foo
	\end{minted}
	
\end{frame}

\begin{frame}{prvalues and void}
\centering
	Prvalues expressions can return void type.
\end{frame}

\begin{frame}[fragile]{Type completeness requirements}
	\centering
	Prvalues expressions that yield type T needs this type to be complete.
	\vskip 0.5em
	\hrule
	\vfill
	
	\begin{minted}{c++}
Foo first_copy_of_two(Foo& first, Foo& second){return first;}
		
int main(){
  // call to first_of_two is now prvalue expression
  // the program will not compile
  const auto& result = first_of_two(second_foo(),
                                    first_foo());
  if(&result == &second_foo())
    std::cout << "result is second" << std::endl; 
}
	\end{minted}
\end{frame}

\section{Expression categories conversion}

\begin{frame}{Types of categories conversions}
	test
\end{frame}

\end{document}